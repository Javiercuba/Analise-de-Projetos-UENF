% Prof. Dr. Ausberto S. Castro Vera
% UENF - CCT - LCMAT - Curso de Ci\^{e}ncia da Computa\c{c}\~{a}o
% Campos, RJ,  2022
% Disciplina: An\'{a}lise e Projeto de Sistemas
% Aluno: 

\chapterimage{planejamento.png} % Table of contents heading image
\chapter{Etapa de Planejamento}


Neste cap\'{\i}tulo \'{e} apresentado os detalhes sobre a implementação
do sistema junto com suas etapas e seu valor agregado.



\section{Solicita\c{c}\~{a}o do Sistema}

\begin{itemize}
       \item Responsavel do sistema: Javier Ernesto Lopez Del Real
       \item Necessidades da empresa
       \begin{itemize}
              \item Necessidade de praticidade da comunicação de informação entre as agências e análise geral dos dados de forma eficaz
       \end{itemize}
       \item Requisitos do negocio:
       \begin{itemize}
              \item Integração entre companhias de diferentes aeroportos, controle geral de faturamento de vendas e controle de clientes e aeronaves, junto com a análise de dado realizado com todas essas entidades que fazem parte do sistema
       \end{itemize}
       \item Valor agregado
       \begin{itemize}
              \item O sistema realiza indiretamente a economia de tempo gasto da equipe, com a análise dos dados será possível controlar melhor as operações identificando processos que precisam ser ajustados gerando inteligência no negócio para tomar melhores decisões. 
       \end{itemize}
       \item Outras informações
       \begin{itemize}
              \item Com a implementação do sistema haverá o aumento
               numero de passagens vendidas e uma melhor logistica das aeronaves
               em cada aeroporto.
              
       \end{itemize}
\end{itemize}


\section{Custos: Desenvolvimento e Operacional}
%%%%%%%%%%%%%%%%%%%%%%%%%%%%%%%


\section{Benef\'{\i}cios}
%%%%%%%%%%%%%%%%%%%%%%%%%%%%%%%


       \subsection{Benef\'{\i}cios Tang\'{\i}veis}


       \subsection{Benef\'{\i}cios Intang\'{\i}veis}


\section{An\'{a}lise de Custos e Benef\'{\i}cios}
%%%%%%%%%%%%%%%%%%%%%%%%%%%%%%%





\section{Estudo de Viabilidade}
%%%%%%%%%%%%%%%%%%%%%%%%%%%%%%%


       \subsection{Calend\'{a}rio }

       \subsection{Cronograma }

       \subsection{Alternativas Tecnol\'{o}gicas }
        Hardware, Software, Treinamento, etc...

       \subsection{Or\c{c}amento }
       Considere as Alternativas Tecnol\'{o}gicas para fazer pelo menos 3 or\c{c}amentos diferentes



       \subsection{Resumo e Recomenda\c{c}\~{o}es}

       Considerando .............. o sistema a ser desenvolvido SIM/N\~{A}O \'{e} vi\'{a}vel do ponto de vista ...................
