% Prof. Dr. Ausberto S. Castro Vera
% UENF - CCT - LCMAT - Curso de Ci\^{e}ncia da Computa\c{c}\~{a}o
% Campos, RJ,  2022
% Disciplina: An\'{a}lise e Projeto de Sistemas
% Aluno:


\chapterimage{sistemas.png} % Table of contents heading image
\chapter{ Introdu\c{c}\~{a}o}

\textit{An\'{a}lise e Projeto de Sistemas} \'{e} uma disciplina orientada a descrever as duas primeiras etapas do Ciclo de Vida de Desenvolvimento de um Sistema (CVDS), neste caso, um sistema computacional.  As refer\^{e}ncias bibliogr\'{a}ficas b\'{a}sicas a serem consultadas s\~{a}o: \cite{Dennis2014}, \cite{Dennis2019} \cite{Gane1983} e \cite{Sommerville2011}. Como bibliografia complementar ser\~{a}o considerados: \cite{Satzinger2012}, \cite{Shelly2012}, \cite{Valacich2020}, \cite{Kendall2020}, \cite{Budgen2021} e \cite{Engholm2013}.

Neste documento apresentamos, passo a passo,  as atividades relacionadas com a An\'{a}lise e Design do sistema....


\section{Descri\c{c}\~{a}o do Sistema Computacional a desenvolver}

\subsection{Sistemas de integração}

\subsection{Controle de vendas}

\subsection{Controle de aeronaves}


\section{Identificando as componentes do meu sistema}
A seguir será apresentado os equipamentos essenciais para o funcionamento do sistema, além de
mostrar como será feito a logística de treinamento para os funcionários que usufruirão do novo
sistema.

\subsection{Componente: Hardware}
\begin{itemize}
       \item Monitores
             \begin{itemize}
                    \item Balcão
                    \item Setor Administrativo
             \end{itemize}
             
       \item Impressoras
             \begin{itemize}
                    \item Balcão de embarque
             \end{itemize}
       \item Microcomputadores
             \begin{itemize}
                    \item Setor administrativo
             \end{itemize}
       \item Servidor
\end{itemize}

\subsection{Componente: Software}
\begin{itemize}
       \item Sistema
             \begin{itemize}
                    \item Sistema de passagens
                    \item Controle de aeronaves
                    \item Gerenciamento dos clientes
                    \item Sistema de Integração entre agencias
                    \item Sistema de exibição de relatorios
             \end{itemize}
\end{itemize}



\subsection{Componente: Pessoas}
\begin{itemize}
       \item Programadoes
       \item Clientes
       \item Analista de sistemas
       \item Analista de dados
       \item Funcionarios administrativos
       \item Funcionarios de venda
       \item Piloto
\end{itemize}

\subsection{Componente: Banco de Dados}

\begin{itemize}
       \item Gerenciador de banco de dados
       \item Passagens
       \item Avioes
       \item Clientes
       \item Viagem
       \item Aeroportos
\end{itemize}

\subsection{Componente: Documentos }
\begin{itemize}
       \item Passagens
       \item Relatorios
       \item Vendas
       \item Manual do sistema
\end{itemize}

\subsection{Componente: Metodologias ou Procedimentos}
\begin{itemize}
       \item Levantamento de requisitos
              \begin{itemize}
                     \item Listagem de todos os equipamentos que serão necessários para o funcionamento do
                     sistema
                     
              \end{itemize}
       \item Analise
              \begin{itemize}
                     \item Estudo de caso de cada partição dos sistemas
                     \item Diagrama para integração entre agencias
                     \item Diagrama do banco de dados
                     \item Estudo de tecnoligias utilizadas
              \end{itemize}
       \item Implementação
              \begin{itemize}
                     \item Desenvolvimento dos aplicativos(sistema interno, website, app mobile)
                     \item Desenvolvimento do banco de dados
              \end{itemize}
       \item Testes
              \begin{itemize}
                     \item Verificação e autenticação de todas as entidades no sistema
                     \item Teste para verificar falhas no sistema 
              \end{itemize}
       \item Treinamento 
       \begin{itemize}
              \item Treinamento para a utilização correta do sistema interno.
              \item Cada setor tera seu treinamento em função do tipo.
       \end{itemize}
\end{itemize}

\subsection{Componente: Mobilidade}
\begin{itemize}
       \item Aplicativo
       \item Celular
       \item Website
       \item Computador
\end{itemize}

\subsection{Componente: Nuvem}
\begin{itemize}
       \item Banco de dados
       \item Sistema
\end{itemize}



