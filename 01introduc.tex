% Prof. Dr. Ausberto S. Castro Vera
% UENF - CCT - LCMAT - Curso de Ci\^{e}ncia da Computa\c{c}\~{a}o
% Campos, RJ,  2022
% Disciplina: An\'{a}lise e Projeto de Sistemas
% Aluno:


\chapterimage{sistemas.png} % Table of contents heading image
\chapter{ Introdu\c{c}\~{a}o}

O sistema a ser desenvolvido tem como objetivo a integração entre
companhias aereas, onde será
apresentado todas as etapa de planejamento. 
Neste capitulo será feita uma descrição do projeto
a ser desenvolvido.

\section{Descri\c{c}\~{a}o do Sistema Computacional a desenvolver}
O projeto será desenvolvidos em tres partes, onde o primeiro é fazer o sistema para fazer a integração 
e comunicação de todas as companhias com a funcionalidade de deixar tudo mais otimizado e prático para
qualquer tipo de manutenção e atualização futura.

A segunda parte que tem como função de criar os controles de vendas, controle de aeronaves, 
controle de gastos, analise dos perfis dos clientes e tomar decisões melhores para a companhia.

A terceira Parte consiste na implementação de um dashboard para analise de dados, com informações 
para controlar melhor a operação.

\subsection{Sistemas de integração}
A principal do sistema tem como objetivo integrar todas a 
filiais da companhia aerea de todo mundo, ele será 
feito com a função de deixar tudo mais simples e prático.
facilitando qualquer manutenção futura ou modificação.

A integração será realizada em todas as partes do
sistema permitindo uma futura analise gráfica
e automatização dentro dessa analise.


\subsection{Controle de vendas}
O controle de vendas será realizado com as funcionalidades 
de vendas de passagens dentro do sistema, integrado com todas
as agencias. 

A partir dessas informações recebidas dentro do 
sistema de controle de vendas será possivel fazer um 
estudo de caso e analise para melhor verificaçãos e validação
das variações dos clientes.

\subsection{Controle de aeronaves}


\section{Identificando as componentes do meu sistema}
A seguir será apresentado os equipamentos essenciais para o funcionamento do sistema, além de
mostrar como será feito a logística de treinamento para os funcionários que usufruirão do novo
sistema.

\subsection{Componente: Hardware}
\begin{itemize}
       \item Monitores
             \begin{itemize}
                    \item Balcão
                    \item Setor Administrativo
             \end{itemize}
             
       \item Impressoras
             \begin{itemize}
                    \item Balcão de embarque
             \end{itemize}
       \item Microcomputadores
             \begin{itemize}
                    \item Setor administrativo
             \end{itemize}
       \item Cameras de segurança
       \item Servidor
\end{itemize}

\subsection{Componente: Software}
\begin{itemize}
       \item Sistema
             \begin{itemize}
                    \item Sistema de passagens
                    \item Controle de aeronaves
                    \item Gerenciamento dos clientes
                    \item Sistema de Integração entre agencias
                    \item Sistema de exibição de relatorios
             \end{itemize}
\end{itemize}



\subsection{Componente: Pessoas}
\begin{itemize}
       \item Programadoes
       \item Clientes
       \item Analista de sistemas
       \item Analista de dados
       \item Funcionarios administrativos
       \item Funcionarios de venda
       \item Piloto
\end{itemize}

\subsection{Componente: Banco de Dados}

\begin{itemize}
       \item Gerenciador de banco de dados
       \item Passagens
       \item Avioes
       \item Clientes
       \item Viagem
       \item Aeroportos
\end{itemize}

\subsection{Componente: Documentos }
\begin{itemize}
       \item Passagens
       \item Relatorios
       \item Vendas
       \item Manual do sistema
\end{itemize}

\subsection{Componente: Metodologias ou Procedimentos}
\begin{itemize}
       \item Levantamento de requisitos
             \begin{itemize}
                    \item Listagem de todos os equipamentos que serão necessários para o funcionamento do
                          sistema
                          
             \end{itemize}
       \item Analise
             \begin{itemize}
                    \item Estudo de caso de cada partição dos sistemas
                    \item Diagrama para integração entre agencias
                    \item Diagrama do banco de dados
                    \item Estudo de tecnoligias utilizadas
             \end{itemize}
       \item Implementação
             \begin{itemize}
                    \item Desenvolvimento dos aplicativos(sistema interno, website, app mobile)
                    \item Desenvolvimento do banco de dados
             \end{itemize}
       \item Testes
             \begin{itemize}
                    \item Verificação e autenticação de todas as entidades no sistema
                    \item Teste para verificar falhas no sistema
             \end{itemize}
       \item Treinamento
             \begin{itemize}
                    \item Treinamento para a utilização correta do sistema interno.
                    \item Cada setor tera seu treinamento em função do tipo.
             \end{itemize}
\end{itemize}

\subsection{Componente: Mobilidade}
\begin{itemize}
       \item Aplicativo
       \item Celular
       \item Website
       \item Computador
\end{itemize}

\subsection{Componente: Nuvem}
\begin{itemize}
       \item Banco de dados
       \item Sistema
\end{itemize}



